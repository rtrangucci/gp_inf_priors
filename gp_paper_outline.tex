\documentclass{article}

% if you need to pass options to natbib, use, e.g.:
% \PassOptionsToPackage{numbers, compress}{natbib}
% before loading nips_2016
%
% to avoid loading the natbib package, add option nonatbib:
%\usepackage[nonatbib]{nips_2016}

%\usepackage{nips}

% to compile a camera-ready version, add the [final] option, e.g.:
\usepackage[final]{nips_2016}

\usepackage[utf8]{inputenc} % allow utf-8 input
\usepackage[T1]{fontenc}    % use 8-bit T1 fonts
\usepackage{hyperref}       % hyperlinks
\usepackage{url}            % simple URL typesetting
\usepackage{booktabs}       % professional-quality tables
\usepackage{amsfonts}       % blackboard math symbols
\usepackage{nicefrac}       % compact symbols for 1/2, etc.
\usepackage{microtype}      % microtypography

\usepackage{amssymb}
\usepackage{mathtools}
\usepackage{latexsym}
\usepackage{amsthm}
\usepackage{enumerate}
\usepackage{epsfig}
\usepackage{graphicx}
\usepackage{color}
\usepackage{float}
\usepackage{subfigure}
\usepackage{amsmath}
\usepackage{MnSymbol}
\usepackage{makeidx}
\usepackage{fancyhdr}
\usepackage{relsize}
\pagestyle{fancy}
\usepackage{lastpage}
\usepackage{url}
\usepackage{mathrsfs}

\newcommand{\F}{\ensuremath{\mathcal F}}
\DeclareMathSymbol{\R}{\mathbin}{AMSb}{"52}
\newcommand{\f}{\ensuremath{\mathcal f}}
\newcommand{\C}{\ensuremath{\mathcal C}}
\newcommand{\M}{\ensuremath{\mathcal M}}
\renewcommand{\H}{\ensuremath{\mathcal H}}
\newcommand{\pisys}{\ensuremath{\mathscr{L}}}
\newcommand{\lsys}[1]{\ensuremath{\lambda \lp #1 \rp}}
\newcommand{\A}{\ensuremath{\mathcal A}}
\newcommand{\E}{\ensuremath{\mathcal E}}
\renewcommand{\L}{\ensuremath{\mathcal L}}
\newcommand{\norm}[1]{\ensuremath{\mathcal \| #1 \|}}
\newcommand{\Exp}[1]{\ensuremath{\mathbb{E} \lb #1 \rb}}
\newcommand{\condExp}[2]{\ensuremath{\mathbb{E} \lb #1 | #2 \rb}}
\newcommand{\lp}{\ensuremath{\left(}}
\newcommand{\rp}{\ensuremath{\right)}}
\newcommand{\lb}{\ensuremath{\left[}}
\newcommand{\rb}{\ensuremath{\right]}}
\newcommand{\B}[1]{\ensuremath{\mathcal B\lp #1 \rp}}
\newcommand{\Pset}[1]{\ensuremath{\mathcal P\lp #1 \rp}}
\newcommand{\siga}[1]{\ensuremath{\sigma\lp #1 \rp}}
\newcommand{\Xrv}[1]{\ensuremath{X\lp #1 \rp}}
\newcommand{\Xrvi}[1]{\ensuremath{X \inv \lp #1 \rp}}
\newcommand{\Yrv}[1]{\ensuremath{Y\lp #1 \rp}}
\newcommand{\Prob}[1]{\ensuremath{\Pb\lp #1 \rp}}
\newcommand{\inv}{\ensuremath{^{-1}}}
\newcommand{\iprod}[2]{\ensuremath{\llangle #1, #2 \rrangle}}
\newcommand{\twopartdef}[4]
{
	\left\{
		\begin{array}{ll}
			#1 & \mbox{if } #2 \\
			#3 & \mbox{if } #4
		\end{array}
	\right.
}
\newcommand\independent{\protect\mathpalette{\protect\independenT}{\perp}}
\def\independenT#1#2{\mathrel{\rlap{$#1#2$}\mkern2mu{#1#2}}}

\title{Prior Formulation for Gaussian Process Hyperparameters}

% The \author macro works with any number of authors. There are two
% commands used to separate the names and addresses of multiple
% authors: \And and \AND.
%
% Using \And between authors leaves it to LaTeX to determine where to
% break the lines. Using \AND forces a line break at that point. So,
% if LaTeX puts 3 of 4 authors names on the first line, and the last
% on the second line, try using \AND instead of \And before the third
% author name.

\author{
  Rob Trangucci \\
  Applied Statistics Center\\
  Columbia University\\
  \texttt{robert.trangucci@gmail.com} 
  \and
  \textbf{Michael Betancourt} \\
  Applied Statistics Center \\
  Columbia University \\
  \texttt{betanalpha@gmail.com} 
  \and
  \textbf{Aki Vehtari} \\
  Department of Computer Science \\
  Aalto University \\
  \texttt{aki.vehtari@aalto.fi} 
  \and
  \textbf{Dan Simpson} \\
  Department of Statistics \\
  University of Toronto \\
  \texttt{dp.simpson@gmail.com} 
}

\begin{document}
% \nipsfinalcopy is no longer used

\maketitle

\begin{abstract}
  Gaussian processes (GP) are measures over functions, and as such, can be used
  as a rich prior for latent functions in Bayesian statistical models. Despite
	the 
	
	
	However,
  the extreme flexibility of the prior requires weakly informative priors on GP
  hyperparameters. We develop a principled approach for specifying weakly
  informative priors for length scale hyperparameters that impose soft
  constraints on the space of functions represented by the Gaussian process
  prior.
\end{abstract}


\section{Introduction}

\section{Difficulties inferring hyperparameters in GPs}

\begin{itemize}
  \item 
\end{itemize}

\subsection{Mat\'{e}rn}

\begin{itemize}
  \item We know Matern should be dangerous in asymptotic regime from theory (ridge!)
  \item Matern also resolves ridge in finite data regime
  \item 
\end{itemize}

\subsection{Exp. quad}

\begin{itemize}
  \item 
\end{itemize}

\section{Requirements for prior specs}

\subsection{Length-scale reqs}

Positive support, zero avoiding. Cut off small length-scales, cut off large
length-scales.

\begin{itemize}
  \item Knowledge of the process motivates upper and lower bounds (physical)
  \item Knowledge of design of the sampling scheme for GP (no matter how much data, can't sample below certain length-scales)
  \item Largest and smallest covariate range motivated prior for computational stability
\end{itemize}

\begin{itemize}
  \item Gamma
  \item inv gamma
  \item GIG
  \item log-normal
\end{itemize}

\subsection{Marginal signal scale and noise scale}

Knowledge of the total variance in y and expected noise, or expected signal
variance, shrink towards zero, can have sharp tail be

\section{Experiments}

\subsection{n = 10}

\subsection{n = 300}

\section{Discussion}

Hopefully, shapes of priors don't matter, only the upper and lower bounds. We
see that fat tailed and thin tailed distributions that agree on small
lenghth-scales lead to nearly equivalent posteriors that don't have the ridges.
This is because we have sigma tail prior cutting off the section that part of
rho space.

\section{Conclusion}

In D > 1, GPs present even larger challenges because of the fact that the
number of interpolating functions grows exponentially. Additionally, 
infill becomes that much more out of reach because of the peculiar
problem that most points are far away from one another in high dimensions.


\end{document}
