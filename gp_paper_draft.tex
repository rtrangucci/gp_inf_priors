\documentclass{article}

% if you need to pass options to natbib, use, e.g.:
% \PassOptionsToPackage{numbers, compress}{natbib}
% before loading nips_2016
%
% to avoid loading the natbib package, add option nonatbib:
%\usepackage[nonatbib]{nips_2016}

%\usepackage{nips}

% to compile a camera-ready version, add the [final] option, e.g.:
\usepackage[final]{nips_2016}

\usepackage[utf8]{inputenc} % allow utf-8 input
\usepackage[T1]{fontenc}    % use 8-bit T1 fonts
\usepackage{hyperref}       % hyperlinks
\usepackage{url}            % simple URL typesetting
\usepackage{booktabs}       % professional-quality tables
\usepackage{amsfonts}       % blackboard math symbols
\usepackage{nicefrac}       % compact symbols for 1/2, etc.
\usepackage{microtype}      % microtypography

\usepackage{amssymb}
\usepackage{mathtools}
\usepackage{latexsym}
\usepackage{amsthm}
\usepackage{enumerate}
\usepackage{epsfig}
\usepackage{graphicx}
\usepackage{color}
\usepackage{float}
\usepackage{subfigure}
\usepackage{amsmath}
\usepackage{MnSymbol}
\usepackage{makeidx}
\usepackage{fancyhdr}
\usepackage{relsize}
\pagestyle{fancy}
\usepackage{lastpage}
\usepackage{url}
\usepackage{mathrsfs}

\newcommand{\F}{\ensuremath{\mathcal F}}
\DeclareMathSymbol{\R}{\mathbin}{AMSb}{"52}
\newcommand{\f}{\ensuremath{\mathcal f}}
\newcommand{\C}{\ensuremath{\mathcal C}}
\newcommand{\M}{\ensuremath{\mathcal M}}
\renewcommand{\H}{\ensuremath{\mathcal H}}
\newcommand{\pisys}{\ensuremath{\mathscr{L}}}
\newcommand{\lsys}[1]{\ensuremath{\lambda \lp #1 \rp}}
\newcommand{\A}{\ensuremath{\mathcal A}}
\newcommand{\E}{\ensuremath{\mathcal E}}
\renewcommand{\L}{\ensuremath{\mathcal L}}
\newcommand{\norm}[1]{\ensuremath{\mathcal \| #1 \|}}
\newcommand{\Exp}[1]{\ensuremath{\mathbb{E} \lb #1 \rb}}
\newcommand{\condExp}[2]{\ensuremath{\mathbb{E} \lb #1 | #2 \rb}}
\newcommand{\lp}{\ensuremath{\left(}}
\newcommand{\rp}{\ensuremath{\right)}}
\newcommand{\lb}{\ensuremath{\left[}}
\newcommand{\rb}{\ensuremath{\right]}}
\newcommand{\B}[1]{\ensuremath{\mathcal B\lp #1 \rp}}
\newcommand{\Pset}[1]{\ensuremath{\mathcal P\lp #1 \rp}}
\newcommand{\siga}[1]{\ensuremath{\sigma\lp #1 \rp}}
\newcommand{\Xrv}[1]{\ensuremath{X\lp #1 \rp}}
\newcommand{\Xrvi}[1]{\ensuremath{X \inv \lp #1 \rp}}
\newcommand{\Yrv}[1]{\ensuremath{Y\lp #1 \rp}}
\newcommand{\Prob}[1]{\ensuremath{\Pb\lp #1 \rp}}
\newcommand{\inv}{\ensuremath{^{-1}}}
\newcommand{\iprod}[2]{\ensuremath{\llangle #1, #2 \rrangle}}
\newcommand{\twopartdef}[4]
{
	\left\{
		\begin{array}{ll}
			#1 & \mbox{if } #2 \\
			#3 & \mbox{if } #4
		\end{array}
	\right.
}
\newcommand\independent{\protect\mathpalette{\protect\independenT}{\perp}}
\def\independenT#1#2{\mathrel{\rlap{$#1#2$}\mkern2mu{#1#2}}}

\title{Prior Formulation for Gaussian Process Hyperparameters}

% The \author macro works with any number of authors. There are two
% commands used to separate the names and addresses of multiple
% authors: \And and \AND.
%
% Using \And between authors leaves it to LaTeX to determine where to
% break the lines. Using \AND forces a line break at that point. So,
% if LaTeX puts 3 of 4 authors names on the first line, and the last
% on the second line, try using \AND instead of \And before the third
% author name.

\author{
  Rob Trangucci \\
  Applied Statistics Center\\
  Columbia University\\
  \texttt{robert.trangucci@gmail.com} 
  \and
  \textbf{Michael Betancourt} \\
  Applied Statistics Center \\
  Columbia University \\
  \texttt{betanalpha@gmail.com} 
  \and
  \textbf{Aki Vehtari} \\
  Department of Computer Science \\
  Aalto University \\
  \texttt{aki.vehtari@aalto.fi} 
  \and
  \textbf{Dan Simpson} \\
  Department of Statistics \\
  University of Toronto \\
  \texttt{dp.simpson@gmail.com} 
}

\begin{document}
% \nipsfinalcopy is no longer used

\maketitle

\begin{abstract}
  Gaussian processes (GP) are measures over functions, and as such, can be used
  as a rich prior for latent functions in Bayesian statistical models.
  Unfortunately, marginalizing over hyperparameters in popular GP kernels like
  the Mat\'{e}rn and the exponentiated quadratic is challenging due to the
  asymptotic and pre-asymptotic properties of these kernels. Successfully
  performing full-Bayesian inference on models using GPs as priors requires
  weakly informative priors on length-scale and signal marginal standard
  deviation. After delving into the We develop a principled approach for
  specifying weakly informative priors for length scale hyperparameters that
  impose soft constraints on the space of functions represented by the Gaussian
  process prior.
\end{abstract}


\section{Introduction}

As noted by many authors (\citet{flaxman2015fast},
\citet{stein2012interpolation}, \citet{rasmussen2005gaussian},
\citet{fuglstad2015interpretable}), learning hyperparameters in Gaussian
process (GP) kernels is notoriously hard. The preponderance of literature has
focused estimating hyperparameters using marginal maximum likelihood for point
estimation (\citet{stein2012interpolation}, \citet{rasmussen2005gaussian},
\citet{warnes1987problems}), and the asymptotic properties of posterior
conditional mean functions in GP regressions (\citet{seeger2008information},
\citet{stein2012interpolation}, \citet{rasmussen2005gaussian},
\citet{williams2000upper}) with some exceptions such as
\citet{neal1998regression} and \citet{vanhatalo2013gpstuff}. As applied
statisticians, we face datasets with finitely many observations. As such there
will be many sets of hyperparameters that are consistent with our data;
selecting any one point will lead to overfitting and poor out-of-sample
predictions. Instead, we advocate placing priors over hyperparameters and
integrate over our uncertainty when making predictions for new data. The
empirical advantage of full Bayesian inference versus point estimation are
well-documented (\citet{vehtariloo}). In order to do this we need to engineer
principled priors that regularize the Gaussian process sufficiently to admit
practical fitting without compromising its statistical utility. In this paper
we'll investigate the consequences of priors on hyperaparamters from
theoretical and empirical perspectives, with a focus on the finite data regime.

In section \ref{prior_art} we define the 1-D Gaussian process regression setting,
the hyperparameters of interest, and the mathematical basis for the manifestations
of weak identifiability in kernel hyperparameters. In section \ref{experiment}, we 
outline the experimental set up, and in section \ref{results} we discuss the results 
of our experiments. Stan code is presented in the appendix.

\subsection{Prior art} \label{prior_art}

The typical setting for a Gaussian process prior over a function $f(x)$:
%
\begin{align*}
  \theta & \sim g(\phi) \\
  f(x) & \sim \mathcal{GP}\lp \mu(x),
  K_\theta(x, x) \rp 
\end{align*}
%
where $\text{GP}$ is a stochastic process from which finite-sample realizations are
multivariate Gaussian, and which is completely specified its mean vector $\mu$
and its covariance matrix $K_\theta(x, x)$. The $i, j$ th
element of $K_\theta(x, x)$:
\begin{align*}
  \text{cov}(f(x_i), f(x_j)) = k(x_i, x_j | \theta) 
\end{align*}
We focus on the exponentiated quadratic kernel, and the Mat\'{e}rn 5/2 kernel
where $\theta$ comprises two positive, real-valued hyperparameters, $\ell$, the
length-scale, and $\alpha$, the marginal standard deviation.  The exponentiated
quadratic:
\begin{align} \label{kern_quad}
  k(x_i, x_j | \alpha, \ell) = \alpha^2 
\exp \left(
	- \dfrac{|x_{i} - x_{j}|^2}{2\ell^2}
\right)
\end{align}
%
The Mat\'{e}rn 5/2 kernel:
%
\begin{align} \label{kern_quad}
  k(x_i, x_j | \alpha, \ell) = \alpha^2 
	\left( 
	1 + \dfrac{\sqrt{5} |x_{i} - x_{j}|}{\ell} + \dfrac{5|x_{i} - x_{j}|^2}{3\ell^2}
	\right)
	\exp \left(
	- \dfrac{\sqrt{5}{|x_{i} - x_{j}|}}{\ell} 
\right)
\end{align}
We can connect $f$ to the likelihood in any number of ways, such as using $f$
as a prior for the conditional mean function in a regression setting after
observing paired observations $y, x, \in \{1,\dots,N\}$. We might also use $f$
as a prior for a random intercept parameter in a hierarchical model where there
exists a real, group-level measurement, such as age, or income in a
hierarchical voter turnout model.

When using GP priors in the context of regression and a Gaussian likelihood, $y \sim
\mathcal{N}(f(x), \sigma)$, we can derive the exact posterior density $p\lp f(x) |
\mathbf{y}\rp$, provided $\ell$ and $\alpha$ are fixed. With $f(x)$ and $\mathbf{y}$
jointly normally distributed:
\begin{align*} \begin{pmatrix} f(x)\\ \mathbf{y} \\ \end{pmatrix} \sim
\mathcal{N} \lp \begin{pmatrix} \mathbf{0} \\ \mathbf{0} \end{pmatrix} ,\,\,
  \begin{pmatrix} K_\theta &
  K_\theta  \\ K_\theta &
  K_\theta + \sigma ^ 2 \mathbf{I}\\ \end{pmatrix} \rp
\end{align*}
We can derive $p\lp f(x) | \mathbf{y}\rp$ from the properties of the conditional normal
distribution: 
\begin{align*}
  f(x) \sim
  \mathcal{N}(K_\theta  (K_\theta + \sigma ^ 2 \mathbf{I})^{-1}\mathbf{y},  
  K_\theta - K_\theta (K_\theta + \sigma ^ 2 \mathbf{I})^{-1}K_\theta)
\end{align*}
Out-of-sample predictions are generated similarly:
\begin{align*} 
  \mathbf{\tilde{y}} \sim
  \mathcal{N}(&K_\theta(\mathbf{\tilde{x}},\mathbf{x}) (K_\theta(\mathbf{x},\mathbf{x}) +
  \sigma ^ 2 \mathbf{I})^{-1}\mathbf{y},  
   \\ & K_\theta(\mathbf{\tilde{x}},\mathbf{\tilde{x}})
   -K_\theta(\mathbf{\tilde{x}},\mathbf{x}) (K_\theta(\mathbf{x},\mathbf{x}) +
   \sigma ^ 2 \mathbf{I})^{-1}K_\theta(\mathbf{x},\mathbf{\tilde{x}}) + \sigma
   ^ 2 \mathbf{I})
\end{align*}
However, $\ell$ and $\alpha$ are typically unknown, so we need to infer these
hyperparameters from the data at hand. $\ell$ and $\alpha$
will impact our predictions for $\tilde{\mathbf{y}}$. $\alpha^2$ is the
marginal variance of the stochastic process. For a fixed noise variance,
$\sigma^2$, increasing the marginal variance of the stochastic process
increases the signal-to-noise ratio, $\alpha^2 / \sigma^2$ (SN). $\ell$
controls the process's nonlinearity. 

Naturally, quantifiable characteristics of the GP will involve $\ell$ and $\alpha$ 
For instance, the expected number of crossings of $f(x)$ on the interval $[0,
T]$ at level $u$, $\Exp{C_u}$, are solely functions of $\ell$ and $\alpha$. 
\citet{cramer2004stationary} show that $\Exp{C_u}$:
\begin{align*} 
  \Exp{C_u | \theta} = \dfrac{T}{\pi} 
\sqrt{-\dfrac{k_\theta^{\prime \prime}(0)}{k_\theta(0)}}
  \text{exp}\left(-\dfrac{u^2}{2k_\theta(0)}\right)
\end{align*} 
$\Exp{C_u | \theta}$ for the exponentiated quadratic kernel:
\begin{align*} 
  \Exp{C_u | \alpha, \ell} = \dfrac{T}{\pi \ell}\text{exp}\left(-\dfrac{u^2}{2 \alpha ^ 2} \right)
\end{align*} 
$\Exp{C_u | \theta}$ for the Mat\'{e}rn 5/2 kernel is:
% Add formula
We can see that at $u = 0$, we are left with $\Exp{C_u | \alpha, \ell} = T /
\pi \ell$.  This characteristic is only valid for processes that are twice
mean-square differentiable at zero. Kernels like the Mat\'{e}rn 3/2 and the
exponential kernel (also known as as a Mat\'{e}rn 1/2) have fewer than two
derivatives.  Conditional on having an upcrossing at $x$, if our kernel does
not have a finite second derivative, there are infinitely many upcrossings in
the interval $x + \delta x$.

When concerned with expected numbers of crossings away from zero, $\alpha$ and
$\ell$ are conflated.  This means that we would not be able to identify
$\alpha$ and $\ell$ separately with number of crossings at different values of
$u$.  This is an inevitability of the parameterization of the exponentiated
quadratic kernel, and this feature of GPs can lead to problems when estimating
$\alpha$ and $\ell$ from data as we will soon see.

\subsection{To marginalize or to maximize?}

As noted by \citet{vehtariloo}, there is e:w

\subsection{GPs as priors}

GPs are a good way parameterize priors for random intercepts in a hierarchical
model. When our data can plausibly be generated from a hierarchy we would do
well to encode hierarchical structure into our prior (\citet{}). An
oft-used prior to put on $J$ group-level random intercepts $\theta_j$ is
an exchangable normal model with mean zero and a scale that is inferred from
the data (\citet{gelman2014bayesian}). The generative model is: 
%
\begin{align*}
  \sigma & \sim \mathcal{N}^+(0, s) \\
  \theta_j & \sim \mathcal{N}(\mu, \sigma)
\end{align*}
%
In keeping with the sentiment that all available information should be encoded
in the prior, group-level measurements, $x$, should be encoded into the prior for
group-level intercepts as well. It is intuitively appealing to favor model
configurations that allow for positive correlation between groups that are
similar in measured group-level covariates. Ideally we would learn the 
magnitude of the correlation between group means $\theta_j$ from the
data. GPs allow a principled way to do so. Instead of an exchangable normal
prior for the group-level means, we parameterize our prior:
%
\begin{align*}
  \sigma & \sim \mathcal{N}^+(0, s) \\
  \alpha & \sim p(\phi) \\
  \ell & \sim g(\rho) \\
  \theta & \sim \mathcal{GP}(\mu, K_{\alpha, \ell}(x, x))
\end{align*}
%
where $p(\phi)$ and $g(\rho)$ are densities with support on $\R^+$ and $\phi$
and $\rho$ are the known parameters of these densities.

\subsection{GPs for regression}

The most common setting for a GP is a regression problem, where we endeavour to
learn a conditional mean function $f(x)$ from the data. With a Gaussian
likelihood we have the following model:
%
\begin{align*}
  \sigma & \sim \mathcal{N}^+(0, s) \\
  \beta & \sim p(\rho), \beta \in \R^P \\
  f(x_i) & = \textstyle \sum_{p=1}^P \beta_p \nu_p(x_i) \\
  y_i & \sim \mathcal{N}\lp f(x_i),
  \sigma\rp 
\end{align*}
%
Linear regression limits our exploration to linear functions of the
coefficients, $\beta_p$, and functions, $\nu_p(x)$ of the predictor. In fitting
a sequence of increasingly complex models (\citet{gelman2014bayesian}), the
modeler defines which transformations of the predictor to include in the design
matrix. There is much discretion left to the modeler as to how to build out the
regression model. Ideally we would dispassionately specify the model expansion
process. In fact, an optimal approach might be to allow for all possible
transformations (say, $P \rightarrow \infty$) of the predictor and to learn the
weight afforded to each transformation. GPs allow us to do just that.
%
\begin{align*}
  \sigma & \sim \mathcal{N}^+(0, s) \\
  \alpha & \sim p(\phi) \\
  \ell & \sim g(\rho) \\
  f(x) & \sim \mathcal{GP}(\mu, K_{\alpha, \ell}(x, x)) \\
  y_i & \sim \mathcal{N}\lp f(x_i),
  \sigma\rp  \\
  i \in & \{1, \dots, N\}
\end{align*}
%
The $f(x) \sim \mathcal{GP}(\mu, K_{\alpha, \ell}(x, x))$ above is equivalent to
$f(x_i)  = \textstyle \lim_{P \to \infty} \sum_{p=1}^P \beta_p \xi_p(x_i)$
(Dan, can you expand on what different kernel choices mean about different bases?)
Choosing the kernel $k(x, x)$ that defines $K(x, x)$ will yield sets of 
basis functions. One set of basis functions, $\xi_p$, are the eigenfunctions 
of the covariance kernel (\citet{rasmussen2005gaussian}).

\section{Difficulties inferring hyperparameters in GPs}

GPs allow for princpled priors in rich Bayesian models. They are flexible and
expressive. 

Its flexibility comes at a price, however, which is that in
practice learning $\ell$ and $\alpha$ is hard in some settings, and impossible
in others. This is true for both the exponentiated quadratic and the
Mat\'{e}rn. 

  \subsection{Mat\'{e}rn}

The Mat\'{e}rn kernel is suggested as a good default kernel for many problems
because it allows for varying 

\begin{itemize}
  \item We know Matern should be dangerous in asymptotic regime from theory (ridge!)
  \item talk about theory for Matern nonidentifiability (equivalent measures, infill)
  \item Matern also resolves ridge in finite data regime
\end{itemize}

\subsection{Exp. quad}

\begin{itemize}
  \item Exp quad has weakly consistent estimators for params
  \item Still have finite data problems, and look, we have the ridge!
  \item Infill asymptotics under which weak consistency is proved are far away
  \item Need very fine data to inform full parameter space
  \item We put spectral density point in this section as intuition behind ridge
\end{itemize}

\section{Requirements for prior specs}

\subsection{Length-scale reqs}

Positive support, zero avoiding. Cut off small length-scales, cut off large
length-scales.

\begin{itemize}
  \item Knowledge of the process motivates upper and lower bounds (physical)
  \item Knowledge of design of the sampling scheme for GP (no matter how much data, can't sample below certain length-scales)
  \item Largest and smallest covariate range motivated prior for computational stability
\end{itemize}

\begin{itemize}
  \item Gamma, log-normal
  \item inv gamma
  \item GIG
  \item half-cauchy (We know these are bad, but they're commonly used)
  \item half-normal (We know these are bad, but they're commonly used)
\end{itemize}

Ways to think about informativeness of priors

\begin{itemize}
  \item Effective degrees of freedom (alpha plus rho)
  \item Zero crossings (only exist for 1D for smooth GPs, because rough GPs
    don't have them) 
  \item Higher level upcrossings $\alpha$ $\rho$ interaction
\end{itemize}

\subsection{Marginal signal scale}

Due to ridge, rho / alpha interaction necessitates thinking about prior
for alpha and the joint prior for rho and alhpa.

The light tail on alpha, cuts off big values of rho, which is how the 
priors interact.

\subsection{Likelihood parameters}

Normal -> $\sigma$
NB -> overdispersion

In GP regression where likelihood is normal, we also have ID problems
with alpha interacting with sigma, and this is common and well-understood
in Gaussian noise and Gaussian latent variable models.

Perhaps it would most sense to think of total variance of $Y$ and SN ratio
deriving $\alpha$ and $\sigma$ from these qtys.

\section{Experiments}

\subsection{n = 10}

\subsection{n = 300}

\section{Discussion}

\subsection{Lessons in 1D}

Hopefully, shapes of priors don't matter, only the upper and lower bounds. We
see that fat tailed and thin tailed distributions that agree on small
lenghth-scales lead to nearly equivalent posteriors that don't have the ridges.
This is because we have sigma tail prior cutting off the section that part of
rho space.

\subsection{Extensions to higher dims}

In D > 1, GPs present even larger challenges because of the fact that the
number of interpolating functions grows exponentially. Additionally, 
infill becomes that much more out of reach because of the peculiar
problem that most points are far away from one another in high dimensions.


\bibliographystyle{plainnat}
\bibliography{bib_inf_priors}

\end{document}
